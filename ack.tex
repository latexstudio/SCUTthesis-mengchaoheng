\chapter{致谢}
%
弹指一挥间,三年的研究生学习生活已到尾声。回顾这充实而又令人难忘的三年,我
掌握了很多知识,接触到了很多人和事,我感谢华工给我提供了如此优越的学习环境,
让我在其中不断成长。

在这里,首先我要感谢我的导师裴海龙教授。裴老师不仅给我提供了研究本课题的
机会,而且在课题研究的各个阶段都给我指明了方向。裴老师有着深厚的学术功底和开阔的视野、敏锐的眼光和严谨
的治学态度,身体力行的影响着我朝正确的研究道路前进,使我受益匪浅,值得我终生
学习和敬重。

同时我还要感谢实验室的胡朗星师傅,他的航模飞行技术精湛,为飞行试验保驾护航。他的教学风格风趣幽默,使我受益匪浅。

感谢实验室的同届其他同学:杨少基、区晨希、简昱颖、李明辉、杨鑫、韩世豪、伍期哲,三年来,我从他们身上学到了很多知识。感谢吴伟坊、谢俊文、姚土才、陈宏润、夏义道等师兄在学习和科研上的耐心指导和无私帮助,至今非常怀念跟他们去三水做实验的时光。感谢陈枫、黄霖杰、吕国刚等师弟,是他们的帮助使我顺利完成论文试验。

还要特别感谢程子欢师兄,正是在他的悉心教导下,我从一个对无人机一窍不通的机械专业本科生,变成一个对涵道无人机有极大兴趣的硕士研究生。子欢师兄在飞行器设计、建模及控制等方向有着深入的研究,他时常会亲自帮我推公式、帮我修改论文。正是在他的帮助下,我顺利发表了小论文并完成学位论文。三年来,每一次外场做试验的场景至今仍历历在目:当崭新的飞机缓缓地飞过三水江边,子欢师兄就坐在那里,深情的目光望过去,都是自己日夜奋战的影子。他是师兄,他是领袖,他是传奇,他是龙门之光。纵使千言万语也难以表达他为涵道组做出的杰出贡献,在此,再次向子欢师兄表示最真挚的感谢!

最后感谢家人对我的支持,没有他们的帮助我无法完成学业,是他们成就了今天的我!
~\\

\begin{minipage}[t]{0.945\textwidth}%
	\begin{flushright}
		蒙超恒\\
		\today\\
		华南理工大学(南校区)%可注释删除
		\par\end{flushright}
\end{minipage}

\cleardoublepage 