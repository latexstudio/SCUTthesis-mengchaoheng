\begin{abstractCN}
涵道风扇式无人机是一种冗余配置操纵面的无人飞行器,其控制系统设计需要解决的问题之一是如何将控制律分配到冗余的操纵面中执行,即控制分配问题。在涵道风扇式无人机的控制分配环节,控制律将作为期望力矩,控制分配算法根据期望力矩求解一组操纵面指令,使得所有操纵面产生的驱动力矩尽可能等于期望力矩。对于本文研究的一类涵道风扇式无人机——单涵道无人机和双涵道无人机,其控制分配问题通常使用伪逆法求解,然而伪逆法不能对任意可达的期望力矩都返回容许控制,使冗余的操纵面牺牲了部分控制能力。

在多操纵面飞行器的控制系统设计中,控制分配器和控制器密切相关,控制分配器是控制器的下一环节。本文针对一类涵道风扇式无人机设计了自抗扰控制器进行姿态控制,并在此基础上重点讨论了两种涵道无人机的控制分配问题:

对单涵道无人机,为了既能对所有可达的期望力矩返回容许控制,又能对不可达的期望力矩做进一步的优化,本文提出一种优先级控制分配方法。该方法先对期望力矩进行矢量分解并划分优先级,再求解约束最优化问题得到容许控制。相比于伪逆法,所提出的方法可对更大范围的期望力矩返回容许控制,而且当期望力矩不可达时,可以防止系统因执行器饱和而产生输出耦合。将所提出的方法应用到基于自抗扰控制的涵道风扇式无人机的控制分配中,仿真及飞行试验验证了该方法的有效性。	
	
对双涵道无人机,引入左右涵道风扇的转速差作为新的操纵面,为滚转通道提供额外的滚转力矩。由于涵道底部的气动舵面和风扇使用不同带宽的执行器(数字舵机和无刷直流电机)驱动,且执行器动态不可忽略,为此本文采用动态控制分配求解双涵道无人机的控制分配问题。首先,基于加权最小二乘法,在惩罚函数中对执行器指令的速率进行惩罚。其次,针对传递函数可近似为一阶惯性环节的执行器,利用后补偿方法来补偿执行器指令衰减。仿真结果表明,带执行器动态补偿的动态控制分配方法,可以将不同频率的期望力矩分配到不同带宽的执行器执行,并且可以降低因执行器动态产生的不良影响。	
\end{abstractCN}

\keywordsCN{涵道风扇式无人机;控制分配;自抗扰控制;多操纵面飞行器}

\begin{abstractEN}
The ducted fan UAV is a kind of aircraft with redundant control surfaces. One of the problems to be solved in the design of its control system is how to allocate the control law to the redundant control surfaces, that is, the control allocation problem.In the control allocation of the ducted fan UAV, the control law will be used as the desired moment. The control allocation algorithm solves a set of control surface commands according to the desired moment, so that the moment generated by all the control surfaces is equal to the desired moment as much as possible. For a type of ducted fan drones studied in this paper—single fan ducted UAV and twin ducted UAV, the
control allocation problem is usually solved by using the pseudo-inverse method. However, the pseudo-inverse method cannot return to admissible control  for all the moments in the attainable moment set, making the redundanted control surface sacrifices some control ability. 

In the design of the control system of aircraft with redundant control surfaces, the control allocator is closely related to the controller, and the control allocator is the next link of the controller. This paper designs an active disturbance rejection controler for attitude control of a type of ducted fan UAV, and on this basis, focuses on the control allocation of two types of ducted UAV:

For the single fan ducted UAV, in order to not only return to admissible control  for all the moments in the attainable moment Set, but also to further optimize the unattainable moment, This paper proposes a priority control allocation method to solving this problem. This method first decomposes the desired moments into a sequence of prioritized partitions, and then solves the constrainted optimization problem to obtain admissible control. Compared with the pseudo-inverse method, the proposed method can return admissible controls for a larger range of desired moments, and when the desired moments is unattainable, it can prevent the system into coupling due to actuator saturation. The proposed method is applied to the control allocation of ducted fan UAV. Simulation and experiments verify the effectiveness of the method.	
	
For the Twin ducted fan UAV, the difference in rotation speed of the left and right ducted fans is introduced as a new control surface to provide additional rolling torque for the rolling channel. Because the aerodynamic control vanes and fan are driven by actuators with different bandwidth(digital servos and brushless DC motor),and the actuator dynamics cannot be ignored.dynamic control allocation is used to solve the Control distribution issues. First, based on the weighted least squares method, the rate of the executor instruction is punished in the penalty function. Secondly, for the actuator whose transfer function can be approximated as the first-order inertial link, the post-compensation method is used to compensate the actuator command attenuation. The simulation results show that the dynamic control distribution method with actuator dynamic compensation can distribute the expected torque of different frequencies to actuators with different bandwidths for execution, and can reduce the adverse effects caused by the actuator dynamics.
\end{abstractEN}

\keywordsEN{ducted fan UAV; control allocation; active disturbance rejection control; aircraft with redundant control surfaces}